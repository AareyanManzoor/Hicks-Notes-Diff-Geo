\documentclass[../main]{subfiles}
\begin{document}

\section{Examples}\label{ch02:s5}

See Figure \ref{fig:ch02fig3} for sketches of (\ref{enu:CH02S05.1}), (\ref{enu:CH02S05.2}), (\ref{enu:CH02S05.3}).

\begin{enumerate}
    \item Let $M$ be an $(n-1)$-dimensional hyperplane in $\bR^n$, i.e., let $N = (a_1,\dots,a_n)$ determine a \emph{constant} unit normal field on $M$. Then $L(X) = \connection_XN = (Xa_1,\dots,Xa_n) = 0$ for all $X$ at all points of $M$, i.e., $L \equiv 0$ on all of $M$. Thus $M$ consists entirely of flat (umbilic) points , the total curvature $K$ and mean curvature $H$ (and all others) are identically zero. All the fundamental forms, except the first, are completely singular. Every vector is asymptotic and a direction of curvature, and all principal curvatures are zero.\label{enu:CH02S05.1}
    \item Let $M$ be $S$, the unit sphere about the origin in $\bR^n$, and let $N$ be the outer normal on $S$, i.e., if $p=(a_1,\dots,a_n)$ then $N(p) = (a_1,\dots,a_n)$. Thus the sphere map $\eta$ is the identity map, $\eta_*$ is also the identity map, and hence $L(X) = X$ for all $X$. Thus $K\equiv 1$, $H\equiv (n-1)$ on $S$. All the fundamental forms are equal to the first fundamental form, all points are umbilic, and all principal curvatures are unity. Every vector is a direction of curvature and there are no asymptotic directions. \label{enu:CH02S05.2}
    \item Let $M$ be the cylinder $C = \{(t_1,\dots,t_n)\in\bR^n\colon\sum_{1}^{n-1} (t_i)^2 = 1\}$ with $N =$ the ``outer'' normal. For $X = e_n = (0, 0, \dots, 0, 1)$ we have $LX = 0$, and for $X$ orthogonal to $e_n$ and tangent to $C$ we have $LX = X$. Hence $K \equiv 0$, $H\equiv (n-1)$, all principal curvature are unity except one which is zero, etc.\label{enu:CH02S05.3}
    
\begin{figure}[ht]
    \centering
    \incfig{pieces-of-examples-(1),-(2),-(3)}
    \caption{Pieces of Examples (1), (2), (3)}
    \label{fig:ch02fig3}
\end{figure}

    
    \item Next let $M$ be an open piece of a surface of revolution about the $z=e_3$-axis in $\bR^3$ (vaguely: $M$ is obtained by revolving a $\CInfty$ plane curve about an axis in the plane). Let $P$ be a plane containing the $z$-axis and take $m$ in $M\cap P$ (and let us consider the point $m$ not on the $z$-axis at first). 
    
    Since the normal $N$ lines in $P$, the vector $\covariant_XN = L(X)$ lies in $P$ and is tangent to $M$ so $L(X) = kX$ and $X$ is a direction of curvature, where $X$ is the unit tangent to a meridian curve. From the remarks preceding the examples there is a direction of curvature orthogonal to $X$, so the unit vector $Y$ tangent to the parallel curves is a direction of curvature. The vector field $\covariant_XX$ is zero or orthogonal to $X$ and must lie in the plane $P$, hence $\covariant_XX = \pm\bar{k}_{1}N$, so $\connection_XX = 0$, and we see the meridians are geodesics. If the parallel curve through $m$ is a geodesic, then $\covariant_YY$ is normal to $M$ and not zero, since these curves are not geodesics in $\bR^3$. But $\covariant_YY$ is orthogonal to $e_3$, the $z$ direction, hence a parallel curve is a geodesic on $M$ iff the normal $N$ along the parallel curve is horizontal (i.e., orthogonal to the $z$-axis). If $m$ is a point on the $z$-axis, then every direction $X$ is tangent to a meridian and hence is a direction of curvature, so $m$ is umbilic and $K(m)\ge 0$.\label{enu:CH02S05.4}
    \item Let us apply the analysis of example \ref{enu:CH02S05.4} to a torus, i.e., let $M$ be obtained by rotating a circle $C$ in the $x, z$-plane about the $z$-axis here we assume the circle does not intersect the $z$-axis. 
    
    \subfile{./figures/ch2fig4}
    
    Then the meridians generated by $C$ are geodesic, as is the minimum length parallel $A$ and maximum length parallel $B$. Along $B$, $M$ has positive curvature, along $A$ the curvature is negative, and the curvature is zero on the extreme top and bottom curves $E$ and $F$ where $N$ is constant. Indeed, if $r_1$ is the radius of $A$ and $r_2$ is the radius of $B$, then $a = \dfrac{r_2-r_1}{2}$ is the radius of $C$ and 
    \[ 
    \begin{aligned}
        K &= \frac{1}{ar_2} = \frac{2}{r_2(r_2 - r_1)}\hspace{1em} \text{on } B, \\
        K &= -\frac{1}{ar_1} = \frac{-2}{r_1(r_2-r_1)}\hspace{1em} \text{on } A,
    \end{aligned} 
    \]
    
    These expressions can be derived as follows. Let $X$ be the unit tangent field to a circle of radius $r$ about the origin in $\bR^2$ (see Figure \ref{fig:ch02fig4}) so $f(t) = (r\cos(t/r),r\sin(t/r))$ parameterizes the circle to fit $x$. Then evaluating a unit outer normal $N$ on $f(t)$ gives $N\circ f(t) = (\cos(t/r),\sin(t/r))$. Hence, \[\dv{}{t}(N\circ f(t))= \dfrac{1}{r}X,\] or if the circle lies on the surface then we see $\covariant_XN = L(X) = \dfrac{1}{r}X$. Now apply this to the circles on the torus. \label{enu:CH02S05.5} 
    \item We discuss ruled surfaces and developable surfaces briefly. A \defemph{ruled surface}\index{ruled surface} is a two-dimensional submanifold $M$ of $\bR^3$ such that through each point $p\InText M$ there passes a segment of a straight line (the generator through $p$) which lies in $M$. When the normal field is a parallel field in $\bR^3$ along the generators, (thus the tangent plane is constant along generators), then the ruled surface is a \defemph{developable surface}\index{developable surface}. Notice we only consider the $\CInfty$ case, although the above definitions can be generalized. \label{enu:CH02S05.6}
\end{enumerate}

Let $M$ be a ruled surface, and let $X$ be a $\CInfty$ unit vector tangent to the generator at each point of $M$. The generators are geodescis in $\bR^3$, so $\covariant_XX = 0$, and hence, from the Gauss equation, $\connection_XX = 0$ and $\ip{LX}{X} = 0$ (so generators are asymptomatic lines). Let $Y$ be a unit vector field orthogonal to $X$ in the neighborhood of a point $p$, then $K = \ip{LX}{X} \ip{LY}{Y} - \ip{LX}{Y}^{2} = -\ip{LX}{Y}^{2}\le 0$ in this neighborhood. \emph{Thus a ruled surface has non-positive curvature}. For a \emph{developable surface}, $0 = \covariant_XN = LX$ so $K \equiv 0$. A theorem due to Massey (see Chapter \ref{ch03}) states a closed connected surface is developable iff $K \equiv 0$.

We study the neighborhood of a point $p$ in a ruled surface $M$. Let $f(t)$ be the $\CInfty$ curve through $p$ which is parameterized by arc-length and is orthogonal to the generators at each point. Let $T$ be the tangent to $f$ (say $T = Y$ along $f$), and let $f(0) = p$. Then the map $(t,s)\mapsto f(t)+sX(t)$ gives a coordinate system from a neighborhood of $(0, 0)$ in $\bR^2$ to a neighborhood of $p$ in $M$. 

\subfile{figures/ch2fig5} % Fig 2.5

Let $N$ be a local unit normal for this coordinate neighborhood. The unit fields $X$, $T$, $N$ give an orthonormal frame along $f$, and we next obtain the Frenet formulas for this frame. On $f$ we have 
\[ 1 = \ip{X}{X} = \ip{T}{T} = \ip{N}{N} \text{ so } 0 = T\ip{X}{X} = 2\ip{\covariant_TX}{X} \]
implies $\covariant_TX$ normal to $X$. Similarly, $\covariant_TN$ normal to $N$ and $\covariant_TT$ normal to $T$. Thus we define functions $a(t)$, $b(t)$, $c(t)$ by
\[
\begin{aligned}
    \covariant_TT &= aX +bN\\
    \covariant_TX &= -aT+cN\\
    \covariant_TN &= -bT-cX
\end{aligned}
\]
where $a = \ip{\covariant_TT}{X} = T\ip{T}{X} - \ip{T}{\covariant_TX}=-\ip{T}{\covariant_TX}$, etc. Holding $s$ constant, we get a curve $f_s(t) = f(t) + sX(t)$ on $M$ with tangent
\[ A = T+s\covariant_TX = (1-as)T+scN \]
(note that $T(t)$ and $N(t)$ are vectors at $f(t)$ which are rigidly translated in $\bR^3$ to this $f_s(t)$ to give $A(t)$). The tangent space along a generator is spanned by $A$ and $X$ (and $A$ is orthogonal to $X$), hence this tangent space is $\emph{constant}$ along a generator iff $c=0$. The function $c/(c^2+a^2)$ is called the \emph{distribution parameter} and it is independent of the particular orthogonal trajectory $f$ (which we show later). Thus (a) $M$ is developable, (b) $K=0$, (c) $c=0$, (d) $LX = 0$, (note $\ip{LX}{T} = \ip{LT}{X}-c$), and (e) $\covariant_TX$ is tangent to $M$, are all equivalent for $M$ closed and connected (assuming Massey's theorem). 

Assuming $M$ is closed (and ruled with $c\ne 0$), on each generator there exists a distinguished point called the \emph{central point}, and these points determine the \emph{curve of striction} on the surface. Fixing two generators, say for $t_1 < t_2$, we compute the length $J(s)$ of an orthogonal trajectory between these two generators by
\[
\begin{aligned}
J(s) &= \int_{t_1}^{t_2} \sqrt{\ip{A}{A}}\dd t \\
&= \int_{t_1}^{t_2} \sqrt{1-2as+a^2s^2 + c^2s^2} \dd t
\end{aligned}
\]

Let us find the value of $s$ which minimizes $J(s)$, and we get $J'(s) = 0$ if
\[ -2a+2(a^2+c^2)s= 0\]
or $s = a/(a^2+c^2)$ at $t_1$ as $t_2 \to t_1$. Hence the \emph{curve of stricture} is the curve
\[ f+\frac{a}{a^2+c^2}X \]
as a function of $t$. This is precisely the point on each generator where the tangent plane is normal to $\covariant_TX(t)$ since $\covariant_TX$ is orthogonal to $X$ we know, and\newline $0 = \ip{\covariant_TX}{A} = -a+a^2s+c^2s$ again gives $s=\dfrac{a}{a^2+c^2}$. As a problem we leave the formula for the curvature,
\[ K(t, s) = \frac{-c^2}{(1-2as+a^2s^2+c^2s^2)^2}, \]
and hence the central point on each generator is also characterized as the point where $K$ is a maximum ($\norm{K}$ a minimum). At the central point, \newline $K = -(a^2+c^2)^2/c^2$, which shows the distribution parameter $\dfrac{c}{a^2+c^2}$ depends only on the generator.

If $s=0$ gives the central point on a particular generator, i.e., we take our orthogonal curve $f$ from this central point, then $\covariant_TX$ is normal to $m$ at $s=0$ and $a=0$. Thus the distribution parameter $p=1/c$ and \[K(t,s) = -\dfrac{c^2}{(1+c^2s^2)^2}=-\dfrac{p^2}{(p^2+s^2)^2}\]. Along this generator $A = T+csN$ where $T$ and $N$ are vectors at the central point, hence the normal $N(s)$ along the generator is given by
\[ 
\begin{aligned}
N(s) &= \frac{-scT + N}{\sqrt{1+s^2c^2}} \\
&= \frac{-sT + pN}{\sqrt{p^2+s^2}}
\end{aligned}
\]

Thus if $\phi$ is the angle between the normal $N(s)$ and the normal $N$ at the central point, we have $\tan \phi = s/p$, i.e., \emph{the tangent of $\phi$ is directly proportional to the distance from the central point}. This is Chasles theorem (1839). This also shows the tangent plane turns even through $180^\circ$ along a generator (turning $90^\circ$ on either side of the central point). For references, see \cite[p. 189]{struik1961lectures} and \cite[p. 107]{willmore1959introduction}. 

We point out we could have viewed the ruled surface discussed above as being generated by the curve $f(t)$ and the field $X(t)$ along the curve. To generate surfaces in this way $X$ need not be orthogonal to $T$. Indeed, in case $\covariant_TT\ne 0$, then we generate a surface via $(t,s) \mapsto f(t) + sT(t)$, for small $s> 0$ (or small $s < 0$), which we call the \emph{tangential developable} of the curve $f$, which si the $\emph{edge of regression}$ of these two surfaces. It is a surface, since $A = T+s\covariant_TT$ is independent of $X=T$ (for $s\ne 0$), and the tangent space along a generator will be determined by $T$ and $\covariant_TT$ for all $s$; hence the surface is developable. It is, of course, not a closed surface in general (see. \cite[p. 66]{struik1961lectures}). 
\begin{center}
\subfile{figures/ch2fig6} % Fig 2.6
\end{center}
\end{document}
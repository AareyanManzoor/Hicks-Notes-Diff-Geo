\documentclass[../main]{subfiles}
\begin{document}

\section{The Fundamental Existence Theorem for Hypersurfaces}\label{ch09:s2}

let $U$ be an open set in $\mathbb{R}^n$ on which is defined the real valued $\CInfty$ functions $g_{ij}$ and $b_{ij}$ for $1 \leq i$,  $j \leq n$ such that the matrices $(g_{ij})$ and $(b_{ij})$ are symmetric and $(g_{ij})$ is positive definite. Roughly speaking, we prescribe conditions which imply the existence of a coordinate system on a hypersurface of $\mathbb{R}^{n+1}$ such that the matrices $(g_{ij})$ and $(b_{ij})$ are the coordinate representations of the first and second fundamental forms, respectively. We demand that $(g_{ij})$ and $(b_{ij})$ satisfy the Gauss curvature and Codazzi-Mainardi equations, and explain this demand. On $U$ define functions $\Gamma^i_{jk}$, in terms of the $g_{ij}$ by the classical formula (see section \ref{ch06:s2}) and define functions
\begin{itemize}
    \item $w_{ij}(e_k) = \Gamma^i_{jk},$
    \item $w_{n+1,j}(e_k) = -b_{jk},$
    \item $w_{j,n+1} = \displaystyle \sum_{r=1}^n (g^{-1})_{jr}b_{rk},$
    \item $w_{n+1,n+1}(e_k) = 0$,
\end{itemize}
 for all $i, j, k \leq n$. Then if there was a coordinate system with coordinate fields $e_1 , \dots e_n$ whose image set was $U$, the Gauss curvature equations and Codazzi-Mainardi equations imply (see section \ref{ch06:s6})

\begin{equation}\tag{1}\label{eqn:9.1}
     \dd{w}_{ij}(e_r, e_s) = -\sum_{k = 1}^{n+1}w_{ik} \wedge w_{ks}(e_r, e_s)
\end{equation}
and
\begin{equation}\tag{2}\label{eqn:9.2}
      \dd{w}_{j,n+1}(e_r, e_s) = -\sum_{k = 1}^{n}w_{jk} \wedge w_{k,n+1}(e_r, e_s)
 \end{equation}

respectively. Thus we can say $(g_{ij})$ and $(b_{ij})$ satisfy the Gauss curvature and Codazzi-mainardi equations if \ref{eqn:9.1} and \ref{eqn:9.2} hold for the functions defined on $U$ where the left sides are computed by \[\dd{w}_{ij}(e_r, e_s) = \pdv{}{u_r}w_{ij}(e_s) - \pdv{}{u_s}w_{ij}(e_r),\text{ etc.}\]



\begin{theorem} \label{thm:ch9.2.1} %format thm:chx.x.x
Let ($g_{ij}$) and ($b_{ij}$)  be defined on $U$ as described above and suppose they satisfy the Gauss curvature and Codazzi-Mainardi equations. Then for any point $p \in U$, there is a neighbourhood $V \subset U$ and a $\CInfty$ mapping $F: V \functionMaps \mathbb{R}^{n+1}$ such that $F(V)$ is an $n$-dimensional submanifold of $\mathbb{R}^{n+1}$, $F^{-1}$ is a coordinate map on $F(V)$, and ($g_{ij}$) and ($b_{ij}$) are the coordinate representation matrices of the first and second forms of $F(V)$, respectively.
\end{theorem}

\begin{proof}
let $u_1, \dots , u_n$ be the natural coordinate functions on $U$.
We seek \newline $(n+1)\, \mathbb{R}^{n+1}$-valued functions $e_1 ,\dots, e_{n+1}$ defined on $U$ that satisfy the Gauss equations and Weingarten equations, i.e.,
\[\tag{3}\label{eq:9.3}\pdv{e_i}{u_j} = \sum_{k=1}^{n+1} w_{ki}(e_j)e_k = \big(\covariant_{e_j}(e_i)\big) \]
where $j=1, \ldots, n$ and $i=1, \ldots, n+1$. Each of the equations in (\ref{eq:9.3}) has $n+1$ components, and the differentiation operator $\pdv{}{u_j}$ is applied to each component. In order to apply the Frobenius' theorem we compute $\dfrac{\partial^{2} e_{i}}{\partial u_{k} \partial u_{j}}$, using (\ref{eq:9.3}) to obtain
\begin{align*}
\dfrac{\partial^{2} e_{i}}{\partial u_{k} \partial u_{j}} &= \sum_r \Big(\pdv{w_{ri}(e_j)}{u_k}e_r+w_{ri}(e_j)\pdv{e_r}{u_k}\Big)\\
 &= \sum_r \Big(\pdv{w_{ri}(e_j)}{u_k}e_r +\sum_s w_{ri}(e_j)w_{sr}(e_k)e_s\Big)
\end{align*}
where we sum $r$ and $s$ from $1$ to $n+1$. The integrability conditions are the equation

\[\tag{4}\label{eq:9.4} \pdv{w_{ri}(e_j)}{u_k} +\sum_s w_{si}(e_j)w_{rs}(e_k) = \pdv{w_{ri}(e_k)}{u_j}+\sum_s w_{si}(e_k)w_{rs}(e_j),\]

which follows from (\ref{eqn:9.1}) and (\ref{eqn:9.2}). 

At the origin in $\bR^{n+1}$ we choose initial vectors $e_{1}, \ldots, e_{n+1}$, so
\begin{itemize}
    \item $\ip{e_{i}}{e_{j}}=g_{i j}(p), $
    \item $\ip{e_i}{e_{n+1}} = g_{i,n+1}(p)=0,$
    \item $e_{n+1} = \pdv{}{u_{n+1}}$
\end{itemize}
for $i, j \leq n$ and $e_{1}, \ldots, e_{n+1}$ positively oriented. Applying the Frobenius theorem we obtain a neighborhood $V_{1}$ of $p$ and $(n+1)\, \bR^{n+1}$-valued $\CInfty$ functions $e_{1}, \ldots, e_{n+1}$ that satisfy (\ref{eq:9.3}).

To check that $\ip{e_i}{e_j} = g_{ij} $ and $\ip{e_i}{e_{n+1}}=0$ at all points on $V_{1}$ for $i, j \leq n$, we must again apply the Frobenius theorem. Let $G_{i j}=\ip{e_i}{e_j}$ on $V_{1}$ for $i, j \leq n+1$. Then by (\ref{eq:9.3}) and the product rule we have
\[\tag{5}\label{eq:9.5} \pdv{G_{ij}}{u_k} = \sum_{r=1}^{n+1}[w_{ri}(e_k)G_{rj} + w_{rj}(e_k)G_{ir}]\]
on $V_{1}$. But from the definition of $w_{r i}(e_{k})=\Gamma_{i k}^{r}$ in terms of $g_{i j}$ we find the functions $g_{i j}$ also satisfy (\ref{eq:9.5}) where we define $g_{i,n+1} \equiv \delta_{i, n+1}$. By using (\ref{eqn:9.1}) and (\ref{eqn:9.2}) we verify the system (\ref{eq:9.3}) satisfies the necessary integrability conditions for the Frobenius theorem and since $ G_{i j}(p)=g_{i j}(p)$ we have $G_{i j}=g_{i j}$ on a neighborhood $V_{2}$ on $p .$

Define functions $A_{i j}$ on $V_{2}$ for $i=1, \ldots, n+1$ and $j=1, \ldots, n$ by \newline $e_{j}=(A_{i j}, \ldots, A_{n+1 j})$, and consider the system of equations
\[\tag{6}\label{eq:9.6} \pdv{f_i}{u_j} = A_{ij} \]
Here \[\pdv{A_{ij}}{u_k} = \pdv{A_{ik}}{u_j}\text{ since } \pdv{e_j}{u_k}=\pdv{e_k}{u_j}(\text{for  }\Gamma^i_{jk}=\Gamma^i_{kj}).\] Thus, letting $f_{i}(p)=0$ for $i=1, \ldots, n+1$, we apply the Frobenius theorem again to get $\CInfty$ functions $f_{1}, \ldots, f_{n+1}$ on a neighborhood $V_{3}$ of $p$ with $V_{3} \subset V_{2}$.

Finally, we define $F: V_{3} \functionMaps \bR^{n+1}$ by $F(m)=(f_{1}(m), \ldots, f_{n+1}(m))$ for $m\in V_{3}$. Then $F$ is $\CInfty$ and $F_\ast\Big(\pdv{}{u_j}(m)\Big) = e_i(m)$ for $j=1, \ldots, n$. Thus $F$ is a diffeo of a neighborhood $V$ of $p$ onto its image $F(V)\subset \bR^{n+1}$ and $V \subset V_{3}$. The map $F^{-1}$ is a coordinate system on $F(V)$ with coordinate vectors $e_{1}, \ldots, e_{n}$, so $g_{i j}=\ip{e_i}{e_j}$ and 
\[\ip{Le_i}{e_j} = \bigg\langle\sum_rw_{r,n+1}(e_i)e_r,e_j\bigg\rangle = \sum_r w_{r,n+1}(e_i)g_{rj} = \sum_{r,s} (g^{-1})_{rs}b_{si}g_{rj} = b_{ji}\]
as desired.

\end{proof}

\end{document}
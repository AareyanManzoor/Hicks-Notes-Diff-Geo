\documentclass[../main]{subfiles}
\begin{document}

%someone started this and didn't finish so i (twiceshy) will try and do that

\section{Submanifolds}\label{ch01:s6}

A $\CInfty$ $k$-manifold is a \defemph{submanifold} of a $\CInfty$ $n$-manifold $\xoverline{M}$ if for every point $p$ in $M$ there is a coordinate neighborhood $\xoverline{U}$ of $\xoverline{M}$ with coordinate functions  $\xoverline{x}_1, \dots, \xoverline{x}_n$ such that the set $U = \{m \in \xoverline{U}: \xoverline{x}_{k+1}(m) = \dots = \xoverline{x}_n(m) = 0\}$ is a coordinate neighborhood of $p$ in $M$ with coordinate functions \newline $x_1 = \xoverline{x}_1 |_{U}, \dots ,x_k = \xoverline{x}_k |_{U} $. These coordinate systems are called \defemph{special} or \defemph{adapted} coordinate systems\index{coordinate systems (special)}. 

Notice it is not required that $M \cap \xoverline{U} = U$ so ``slices'' of $M$ may approach other ``slices'' of $M$ in $\xoverline{M}$ (see problem \ref{pro:17}) and hence the topology on $M$ may not be the relative topology. The definition of submanifold implies $M$ is a subset of $\xoverline{M}$ and $k\ge n$. Letting $i:M \functionMaps\xoverline{M}$ be the inclusion map, then $i$ is $\CInfty$ since $\xoverline{x}_j\circ i$ are $\CInfty$ maps for all special coordinate functions. The inclusion map is also an imbedding (see below) since the Jacobian $i_*$ is non-singular, i.e., $i_*\Big(\pdv{}{x_j}\Big)(p) = \pdv{}{\xoverline{x}_j}(p)$ for $j = 1,\dots,k$. In these notes we will identify a tangent vector $X$ in $\tangentspace{M}{p}$ with its image in $\tangentspace{\xoverline{M}}{p}$ unless there is a possibility of confusion (just as we identify $p$ and $i(p)$).

To make some more standard definitions, let $M$ and $\xoverline{M}$ be $\CInfty$ manifolds and let $f$ be a $\CInfty$ map of $M$ into $\xoverline{M}$. If $f_*$ is non-singular (thus $f_*$ has no kernel) at each point $p$ of $M$, then $f$ is called an \defemph{immersion}\index{immersion} of $M$ into $\xoverline{M}$. If in addition, $f$ is univalent, then $f$ is called an \defemph{imbedding}\index{imbedding} of $M$ into $\xoverline{M}$. A subset $M'$ of $\xoverline{M}$ is called an \defemph{immersed submanifold}\index{submanifold} if there exists a manifold $M$ and an immersion $f: M \functionMaps \xoverline{M}$ such that $f(M) = M'$. (Thus an immersion is a ``local imbedding with self-intersections''.) One can verify (problem \ref{pro:17}) that if $f: M \functionMaps \xoverline{M}$ is an imbedding and $M' = f(M)$, then by defining a differentiable structure of $M'$ so $f$ becomes a diffeomorphism, $M'$ becomes a submanifold of $\xoverline{M}$ (see \cite{helgason2012differential}, p.23).

For examples of submanifolds see examples \ref{enu:CH01S01.5},\ref{enu:CH01S01.6} and \ref{enu:CH01S01.7} at the end of section \ref{CH01:s1}.

It is convenient to define a \defemph{base field}\index{base field} on a set $A$ contained in an $n$-manifold to be a set of $n$ vector fields that are independent at each point of $A$. When each field in a base field is $\CInfty$, then the base field is $\CInfty$. Since a set of of coordinate fields is a $\CInfty$ base field on the coordinate domain, we know $\CInfty$ base fields always exist locally. A $\CInfty$ base field does \emph{not} necessarily exist over a whole manifold (consider the $2$-sphere, $S^2$); indeed, the manifold is called \defemph{parallelizable}\index{parallelizable} if it admits a global $\CInfty$ base field.

We now define a concept which we will often use. Let $M$ be a submanifold of $\xoverline{M}$ as described above. An $\xoverline{M}$-\defemph{vector field}\index{vector field (on submanifold)} $Z$ that is $\CInfty$ on $M$ (or $\CInfty$ on an open set $A$ in $M$) is a map that assigns to each $p$ in $M$ (or $p$ in $A$) a vector $Z_p$ in $\xoverline{M}_p$ such that if $X_1, \dots, X_n$ is any $\CInfty$ base field on a neighborhood $\xoverline{U}$ of $p$ and $Z_m = \sum_{i=1} ^n a_i (m)(X_i)_m$ for $m$ in $M \cap \xoverline{U}$ then the real valued functions $a_i$ are $\CInfty$ on $M \cap U$ for all $i$. Notice $Z_p$ is \emph{not} necessarily tangent to $M$. Since the restriction to $M$, of a $\CInfty$ function on $\xoverline{M}$, is a $\CInfty$ function on $M$, it follows that if $Z$ is $\CInfty$ on $\xoverline{M}$ then $Z|_M$ is an $\xoverline{M}$-vector field that is $\CInfty$ on $M$.


\end{document}
\documentclass[../main]{subfiles}
\begin{document}

\chapter{Riemannian Manifolds and Submanifolds}\label{ch06}

The definition of a Riemannian (and a semi-Riemannian) manifold was given in section~\ref{ch02:s1}. A manifold on which one has singled out a specific symmetric and positive definite (or non-singular) $2$-covariant tensor field, called the \defemph{metric tensor}\index{metric tensor}, is a Riemannian (or semi-Riemannian) manifold. In this chapter we generalize the theory of Chapters~\ref{ch02} and~\ref{ch03} in a natural way. Much of the theory applies to semi-Riemannian manifolds and submanifolds, but, in general, we phrase things only in Riemannian terms.

\subfile{ch06s1}
\subfile{ch06s2}
\subfile{ch06s3}
\subfile{ch06s4}
\subfile{ch06s5}
\subfile{ch06s6}
\subfile{ch06s7}
\subfile{ch06s8}
\subfile{ch06p}

\end{document}
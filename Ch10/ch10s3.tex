%READY FOR PROOFREAD

\documentclass[../main]{subfiles}
\begin{document}

\section{Geometric Interpretation of Riemannian Curvature}\label{ch10:s3}

In this section, let:
\begin{itemize}
    \item $M$ be a Riemannian manifold.
    \item $g$ be a geodesic in $M$ with unit tangent $T$.
    \item $A_0$ be a unit vector in $\ts{M}{g(0)}$ which is orthogonal to $T_0$.
    \item $A'$ be the constant vector field on $\ts{M}{g(0)}$ generated by $A_0$.
    \item $\exp = \exp_{g(0)}$.
    \item $A = \exp_*A'$.
    \item $K = \dfrac{\ip{R(T,A)A}{T}}{\ip AA}$ as a function of $t$ along $g$, whenever $A_t\ne0$.
\end{itemize}
We study the relationship between the Riemannian curvature\index{Riemannian curvature} $K(t)$ of the plane section spanned by $A_t$ and $T_t$ and the length of the vector $A_t$. The field $tA$ is used in the computation since it is a Jacobi field.



\begin{lemma} \label{lem:ch10.3.1}
If $tA_t\ne0$, then
\begin{enumerate}[label=(\arabic*)]
    
    \item \label{enu:ch10s3.1}
    $\displaystyle T\norm{tA} = \frac{\ip{\D_TtA}{tA}}{\norm{tA}}
    = \norm A + \frac{t\ip{\D_TA}{A}}{\norm A}$
    
    \item \label{enu:ch10s3.2}
    $\displaystyle T^2\norm{tA} = -\norm{tA}K(t) + H(t)$ where $H(t)\ge0$.
    
    \item \label{enu:ch10s3.3}
    $\displaystyle \norm{A_t} = 1 - K(0)\frac{t^2}{6} + G(t)t^3$ for $t$ in a neighborhood of zero where $G$ is $\CInfty$.
    
\end{enumerate}
\end{lemma}

\begin{proof}
We compute
\[T\norm{tA} = T\sqrt{\ip{tA}{tA}}
= \frac{\ip{\D_TtA}{tA}}{\norm{tA}}
= \frac{\ip{A+t\D_TA}{tA}}{\norm{tA}}
= \norm A + \frac{t\ip{\D_TA}{A}}{\norm A}\]
Thus
\begin{align*}
    T^2\norm{tA} &= \frac{1}{\norm{tA}}\bigg[\ip{\D_T^2tA}{tA} + \ip{\D_TtA}{\D_TtA} - \frac{\ip{\D_TtA}{tA}^2}{\ip{tA}{tA}}\bigg] \\
    &= \frac{1}{\norm{tA}^3}\big[\ip{R(T,tA)T}{tA}\norm{tA}^2 + \norm{\D_TtA}^2\norm{tA}^2 - \ip{\D_TtA}{tA}^2\big] \\
    &= -\norm{tA}K(t) + H(t)
\end{align*}
where
\[H(t) = \frac{1}{\norm{tA}^3}\big[\norm{\D_TtA}^2\norm{tA}^2 - \ip{\D_TtA}{tA}^2\big]\]
The Schwartz inequality implies $H(t)\ge0$. A straightforward computation shows, as $t\to0$,
$H(t)\to0, H'(t)\to0$
since $(\D_TA)_0=0$ (use normal coord.), hence as $t\to0$,
\[\norm{tA}\to0,
\quad T\norm{tA}\to\norm{A_0}=1,
\quad T^2\norm{tA}\to0,
\quad T^3\norm{tA}\to-K(0)\]

Since $A_t$ does not vanish near $t=0$, the function $\norm{A_t}$ is $\CInfty$ at 0, and hence $F(t)=\norm{tA_t}$ admits a representation
\[F(t) = F(0) + F'(0)t + F''(0)\frac{t^2}{2} + F'''(0)\frac{t^3}{6} + G(t)t^4\]
for $t$ in a neighborhood of 0 where $G$ is a $\CInfty$ function on this neighborhood. Substituting the values for the derivatives of $F$ and cancelling a factor of $t$ then gives \ref{enu:ch10s3.3}.
\end{proof}



The following theorem derives its form essentially from some class notes of Ambrose. 



\begin{theorem} \label{thm:ch10.3.2}
If $K(t)\le0$ for $t\in[0,b]$, then $\norm{A_t}\ge\norm{A_0}=1$ for $t\in[0,b]$. Thus if $K\le0$ for all plane sections at all points of $M$, then $M$ has no conjugate points. If $K(0)<0$, then $\norm{A_t}\ge1$ for $t$ near zero, and if $K(0)>0$, then $\norm{A_t}\le1$ for $t$ near zero.
\end{theorem}

\begin{proof}
Let
\[F(t) = \norm{tA_t} - t\norm{A_0}
= \norm{tA_t} - t\]
Then
\[F(0)=0,
\quad F'(0)=0,
\quad F''(t)=T^2\norm{tA_t}\ge0
\quad \text{if} \quad K(t)\le0\]
Applying the Mean Value Theorem twice,
\[F(t) = F'\br{\bar t}t
= F''\big(\bar{\bar t}\big)\bar tt\ge0
\quad \text{where} \quad 0\le\bar{\bar t}\le\bar t\le t\le b\]
Hence $\norm{A_t}\ge1$ for $t\in[0,b]$.

The second sentence of the theorem follows from the first, and the last two sentences follow from (\ref{enu:ch10s3.3}) in the lemma.
\end{proof}



We obtain a geometric interpretation of Riemannian curvature from the following considerations (see Fig.~\ref{fig:ch10fig3}). The vector $A'$ at the point $bT_0\in\ts{M}{g(0)}$ is tangent to the circle $\sigma$ of radius $b$ about the origin which lies in the plane of $A_0$ and $T_0$. Hence $A=\exp_*A'$ is the tangent at $\exp(bT_0)$ to the curve $\exp\circ\sigma$ in $M$. If $b$ is sufficiently small, then $\exp\circ\sigma$ passes through points that are exactly $b$ units distant from $g(0)$. If $\norm{A_b}>\norm{A'}$ then the curve $\exp\circ\sigma$ is ``stretching'' the curve $\sigma$ near $bT_0$ and the geodesics emanating from $g(0)$ that are determined by $\sigma$ are ``spreading out''. A corresponding statement applies to the case $\norm{A_b}<\norm{A'}$.

\subfile{./figures/ch10fig3} %fig 10.3

\end{document}
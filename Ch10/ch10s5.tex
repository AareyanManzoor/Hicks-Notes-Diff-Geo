%READY FOR PROOFREAD

\documentclass[../main]{subfiles}
\begin{document}

\section{Completeness}\label{ch10:s5}

The theorem that follows gives useful criteria for a Riemannian manifold to be complete. The analytic case was first studied by Hopf-Rinow. The approach we give essentially follows de Rham \cite{derham1952sur}.



\begin{theorem} \label{thm:ch10.5.1}\index{complete}
If $M$ is a connected Hausdorff Riemannian manifold, then statements (a) through (d) below are equivalent, and any one of them implies (e).
\begin{enumerate}[label=(\alph*)]

\item The exponential map is everywhere defined on $\tb M$.

\item $M$ is complete with respect to its Riemannian metric.

\item Bounded closed sets in $M$ are compact.

\item The closed balls $\bar B(m,r)$ are compact for one $m\in M$ and all $r>0$.

\item Any two points in $M$ can be joined by a geodesic segment whose length equals the distance between the two points.

\end{enumerate}
\end{theorem}

\begin{proof}
The implications (d) $\implies$ (c) $\implies$ (b) $\implies$ (a) are all simple. We show (a) implies (d) and (e). Fix $m\in M$ and let
\[B_r = B(m,r),
\quad S_r = \bar B(m,r)\]
\[E_r = \brc{p\in S_r: \text{ there is a geodesic segment $\gamma$ from $m$ to $p$ with $\norm\gamma=d(m,p)$}}\]
We show $E_r$ is compact and $E_r=S_r$ for all $r$, which proves (d) and (e).


\begin{lemma} \label{lem:ch10.5.1a}
The set $E_r$ is compact for all $r$.
\end{lemma}

\begin{proof}
Fix $r$ and let $(m_k)$ be a sequence of points in $E_r$. By (a) there exist points $X_k\in\ts Mm$ such that $\exp_mX_k=m_k$ for all $k$. This follows since a geodesic can always be written as a composite map which is the exponential of a ray in a tangent space. Then $\norm{X_k}<r$ for all $k$, hence $(X_k)$ is a sequence of points in the compact set $\bar B(0,r)$ in the Euclidean space $\ts Mm$. Thus we obtain a subsequence (which we reindex if necessary) $(X_k)$ that converges to $X\in\ts Mm$ with $\norm X\le r$. The corresponding subsequence $(m_k)$ converges to $\exp_mX$, which lies in $E_r$ since $\exp_m$ is $\CInfty$.
\end{proof}


\begin{lemma} \label{lem:ch10.5.1b}
If $E_r=S_r$ for a fixed $r$ and $d(m,p)>r$ then there is a point $\bar m$ such that $d(m,\bar m)=r$ and $d(m,p)=r+d(\bar m,p)$.
\end{lemma}

\begin{proof}
For each integer $k>0$, let $\gamma_k$ be a broken $\CInfty$ curve from $m$ to $p$ with $\norm{\gamma_k}<d(m,p)+\frac1k$. Let $m_k$ be the last point on each $\gamma_k$ that lies in $S_r$, so $d(m,m_k)=r$. Since $S_r$ is compact, the sequence $(m_k)$ has a limit point $\bar m$ and $d(m,\bar m)=r$. But
\[d(m_k,p) \le \norm{\gamma_k}_{m_k}^p
= \norm{\gamma_k}_m^p - \norm{\gamma_k}_m^{m_k}
\le \norm{\gamma_k} - r
< d(m,p) + \frac1k - r\]
Hence $d(\bar m,p)\le d(m,p)-r$, and the triangle inequality proves the opposite inequality.
\end{proof}


\begin{lemma} \label{lem:ch10.5.1c}
For $r\ge0,~E_r=S_r$.
\end{lemma}

\begin{proof}
The proof uses a continuous induction argument on $r$. By definition, $E_r\subset S_r$ for all $r$. For $r=0,~E_0=S_0$. If $E_r=S_r$, then trivially $E_{r'}=S_{r'}$ for all $r'<r$. Conversely, if $E_{r'}=S_{r'}$ for all $r'<r$, then $E_r=S_r$. This follows by taking any point $p\in S_r$ and then choosing $(p_k)\to p$ such that each $p_k$ is in some $S_{r'}$ for $r'<r$. Hence $p_k\in E_{r'}\subset E_r$, and $E_r$ is compact, which implies the limit $p$ is in $E_r$.

Finally, if $E_r=S_r$, then there exists $\ep>0$ such that $E_{r+\ep}=S_{r+\ep}$. Since $S_r$ is compact, we obtain a number $2\ep>0$ such that, for all $p\in S_r$, the map $\exp_p$ is a diffeo from $\brc{X\in\ts Mp: \norm X<2\ep}$ onto $B(p,2\ep)$. Take $p\in S_{r+\ep}$. By lemma \ref{lem:ch10.5.1b}, there is a point $\bar m$ with $d(m,\bar m)=r$ and 
\[d(\bar m,p) = d(m,p) - d(m,\bar m)
\le r+\ep-r \le \ep\]
Hence there is a geodesic segment $\gamma_1$ from $m$ to $\bar m$ with $\norm{\gamma_1}=r$, and a geodesic segment $\gamma_2$ from $\bar m$ to $p$ with $\norm{\gamma_2}=d(\bar m,p)$. Joining $\gamma_1$ and $\gamma_2$ gives a broken $\CInfty$ curve $\gamma$ from $m$ to $p$ with $\norm\gamma=d(m,p)$. Parameterizing $\gamma$ by arc length, there can be no breaks in $\gamma$, so $\gamma$ is a geodesic. Thus $p\in E_{r+\ep}$.
\end{proof}

\end{proof}



We can now prove a classical theorem which illustrates how assumptions about the Riemannian curvature can affect the topology of a manifold.



\begin{theorem}[of Bonnet] \label{thm:ch10.5.2}
If $M$ is a complete connected Riemannian manifold with Riemannian curvature greater than or equal to some $K>0$, then $M$ is compact and its diameter is less than or equal to $\frac\pi{\sqrt K}$.
\end{theorem}

\begin{proof}
We show on every geodesic $g$ there is a conjugate point of $g(0)$ on $\sbr{0,\frac\pi{\sqrt K}}$. If $m\in M$, then by completeness, every point $p\in M$ can be joined to $m$ by a geodesic segment whose length is $d(p,m)$. By Theorem \ref{thm:ch10.2.10}, this geodesic has no conjugate point of $m$ before $p$, hence $d(m,p)\le\dfrac\pi{\sqrt K}$.

Let $g$ be a geodesic with unit tangent $T$, $g(0)=m$, and let $e$ be a unit parallel field along $g$ which is orthogonal to $T$. Let $W_t=(\sin\sqrt Kt)e_t$. Then $W$ is orthogonal to $T$, $W$ vanishes at 0 and $\dfrac\pi{\sqrt K}$, and
\begin{align*}
\D_TW &= (\sqrt K\cos\sqrt Kt)e_t \\
L_W''(0) &= \int_0^{\pi/\sqrt K}\sbr{\ip{R(W,T)W}{T}+\ip{\D_TW}{\D_TW}}\dd t \\
&= \int_0^{\pi/\sqrt K}\sbr{-K(t)\sin^2\sqrt Kt+K\cos^2\sqrt Kt}\dd t \\
&\le K\int_0^{\pi/\sqrt K}\sbr{\cos^2\sqrt Kt-\sin^2\sqrt Kt}\dd t \\
&=0
\end{align*}
where $K(t)=\ip{R(e,t)T}{e}$. If the interval $\Big[0,\dfrac\pi{\sqrt K}\Big]$ was free of conjugate points, then by lemma \ref{lem:ch10.2.9},
\[L_W''(0) > L_Z''(0) = 0\]
where $Z=0$ is the unique Jacobi field along $g$, which coincides with $W$ at 0 and $\dfrac\pi{\sqrt K}$. This contradiction proves the theorem.
\end{proof}



The following theorem, due to K. Nomizu and H. Ozeki, settles the question of the existence of complete Riemannian metrics on a paracompact (or Riemannian) manifold. A Riemannian metric is bounded if the manifold is bounded with respect to the induced metric function.



\begin{theorem} \label{thm:ch10.5.3}
Let $M$ be a connected Hausdorff $\CInfty$ manifold. If $G$ is any Riemannian metric on $M$, then there exist Riemannian metrics $G_1$ and $G_2$, both conformal to $G$, with $G_1$ complete and $G_2$ bounded.
\end{theorem}

\begin{proof}
Since there is more than one Riemannian metric involved, write $G_i(X,Y)$ rather than $\ip XY_i$ for the metric tensor applied to a pair of vectors, $d_i$ for the metric, and $B_i(m,r)$ for the corresponding $r$-ball neighborhoods.

Using the metric $G$, for each $p\in M$, let
\[r(p) = \sup\brc{r: \bar B(p,r) \text{ is compact}}\]
If $r(p)=\infty$ for some $p$, then $G$ is complete by theorem \ref{thm:ch10.5.1}. Suppose $r(p)<\infty$ for all $p$, and we construct $G_1$.

Notice $\norm{r(p)-r(m)}\le d(p,m)$ for all $p$ and $m$, for if $r(p)>r(m)+d(p,m)$, one could increase $r(m)$; hence $r(p)\le r(m)+d(p,m)$ for all $p$ and $m$, and the inequality follows. This proves $r$ is continuous.

Since $M$ is paracompact, it is easy to show there exists $f\in\CInfty(M,\bR)$ with $f(p)>\dfrac{1}{r(p)}$ for all $p$. Let 
\[G_1(X,Y) = f^2(m)G(X,Y)\]
for $X,Y\in\ts Mm$, which defines a $\CInfty$ Riemannian metric $G_1$ on $M$.

That $G_1$ is complete will follow by showing $B(p,\tfrac13)\subset B(p,\tfrac{r(p)}{2})$ and hence $\bar B_1(p,\tfrac16)$ is compact. This implies every Cauchy sequence in the $G_1$ metric must converge. To show this, take $p\in M$ and take $m$ such that $d(p,m)\ge\frac{r(p)}{2}$. Let $\gamma$ ba a broken $\CInfty$ curve from $p$ to $m$, which is parameterized by $G$-arc length, i.e., if $T$ is the tangent to $\gamma$, then $G(T,T)=1$ and $\gamma$ is defined on $[0,L]$ where $L$ is the $G$-length of $\gamma$, so $L\ge\frac{r(p)}{2}$. Letting $L_1$ be the $G_1$-length of $\gamma$,
\[L_1 = \int_0^L\sqrt{G_1(T,T)}\dd t
= \int_0^L(f\circ\gamma)\dd t
= f(\bar p)L
> \frac{L}{r(\bar p)}\]
where $\bar p$ is on $\gamma$ between $p$ and $m$. But
\[\norm{r(\bar p)-r(p)}\le d(p,\bar p)\le L\]
hence
\[r(\bar p)\le r(p)+L,
\quad L'>\frac{L}{r(p+L)}>\frac{L}{3L}=\frac13\]
Hence $d_1(p,m)\ge\frac13$, so $B_1(p,\tfrac13)\subset B(p,\tfrac{r(p)}{2})$.

For the second part of the theorem we may assume $G=G_1$ is complete. Fix a point $m\in M$ and let $f\in\CInfty(M,\bR)$ such that $f(p)>d(m,p)$ for all $p$. Let
\[G_2=e^{-2f}G\]
and we show $G_2$ is bounded. Take $p\in M$ and let $\gamma$ be a geodesic from $m$ to $p$ with tangent $T$ such that $G(T,T)=1$, $\gamma$ is defined on $[0,L]$, and $L=d(m,p)$. Then
\[f\circ\gamma(t)>d(m,\gamma(t))=t\]
for all $t$. Letting $L_2$ be the $G_2$-length of $\gamma$,
\[L_2 = \int_0^L\sqrt{G_2(T,T)}\dd t 
= \int_0^Le^{-f}\dd t
< \int_0^Le^{-t}\dd t
<\int_0^\infty e^{-t}\dd t
= 1\]
Hene $d_2(m,p)<1$ for all $m$ and $p$.
\end{proof}



\begin{corollary} \label{cor:ch10.5.4}
Every Riemannian metric on a manifold is complete iff the manifold is compact.
\end{corollary}



For further work on completeness, see the papers of J. A. Wolf and P. A. Griffiths.

\end{document}
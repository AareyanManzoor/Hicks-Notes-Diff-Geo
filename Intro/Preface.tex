\documentclass[main]{subfiles}

%wordslinger, done
\begin{document}
\chapter*{Preface}
\addcontentsline{toc}{chapter}{Preface}

The following paragraph presents a very brief history of differential geometry and the notation used in these notes. 

Differential geometry is probably as old as any mathematical discipline and certainly was well launched after Newton and Leibniz had laid the foundations of calculus. Many results concerning surfaces in 3-space were obtained by Gauss in the first half of the nineteenth century, and in 1854 Riemann laid the foundations for a more abstract approach. At the end of that century, Levi-Civita and Ricci developed the concept of parallel translation in the classical language of tensors. This approach received a tremendous impetus from Einstein's work on relativity. During the early years of this century, E. Cartan initiated research and methods that were independent of a particular coordinate system (invariant methods). Chevalley's book \cite{chevalley1946theory} continued the clarification of concepts and notation, and it has had a remarkable effect on the current situation. The complete global synthesis of Cartan's approach was achieved when Ehresmann formulated a connexion\footnote{Editors' note: \emph{connexion} means \emph{connection} here and in the rest of the text. The latter is more common in modern usage, while the former likely comes from the corresponding French term. We have decided nonetheless to keep its use throughout the book.} in terms of a fiber bundle. These notes utilize an invariant local method formulated by Koszul. %I propose this footnote regarding the use of "connexion", which we discussed in the server --deri

The first three chapters of this book provide a short course on classical differential geometry and could be used at the junior level with a little outside reading in linear algebra and advanced calculus. The first six chapters can be used for a one-semester course in differential geometry at the senior-graduate level. Such a course would cover the main topics of classical differential geometry (except for the material in chapter~\ref{ch08}) using modern language and techniques, and it would prepare a student for further study in the books of Helgason, Lang, Sternberg, etc. (see list in following paragraph). The entire book can be covered in a full year course. A selection of chapters could make up a ``topics'' course or a course on Riemannian geometry. For example, a course on manifolds and connexions could consist of chapters \ref{ch01}, \ref{ch04}, \ref{ch05}, \ref{ch07} and sections \ref{ch09:s1}, \ref{ch09:s3} and \ref{ch09:s4}. The reader with a little experience should move through the first three chapters fairly quickly.  

The problems are of several types: (a) those that provide explicit computations to test the understanding of the theory, (b) those that require the student to prove theorems similar to those in the text, (c) those that lead the student through supplementary material, some of which may be an integral part of the exposition, and (d) those that lead the student to books or papers in the literature. An introduction to bundle theory and the theory of Lie groups is covered via problem material. Our hope is to give the reader a solid understanding of the basic concepts and to stimulate him to further reading and thinking in differential geometry.

Besides the specific references found in the notes, we would like to mention the following general references: 
\begin{itemize}
    \item Point set topology: \cite{kelley2017general}, \cite{hocking2012topology} and \cite{pervin2014foundations}.
    \item Linear algebra: \cite{halmos2017finite} and \cite{jacobson2013lectures}
    \item Advanced calculus: \cite{buck2003advanced}, \cite{kaplan2003advanced} and \cite{nickerson2013advanced}
    \item Classical differential geometry: \cite{eisenhart2015introduction}, \cite{hilbert1999geometry} and \cite{struik1961lectures}.
    \item Contemporary differential geometry: \cite{auslander2012introduction}, \cite{bishop2011geometry}, \cite{guggenheimer2012differential}, \cite{helgason2012differential}, \cite{kobayashi1963foundations}, \cite{lang2014introduction}, \cite{kobayashi1963foundations} and \cite{sternberg1964lectures} 
    \item History of differential geometry: \cite{struik1961lectures} and \cite{veblen1960foundations}
\end{itemize}

We will use the following conventions: ``iff'' for ``if and only if''; ``$\square$'' for ``Q.E.D.''; Cartan$^3$ will refer to the third reference in the bibliography under Cartan, and when there is only one reference for an author, we omit the superscript 1; $\sum^n_{i=1}$, $\sum_i$, $\sum$ will all be used to indicate a sum is to be made, and in the latter two cases, we hope the omitted information (range or index of summation) is clear from the context.

At this time I would like to express my gratitude to former teachers N. Schwid and V.J. Varineau for their early encouragement, to Miss Margaret M. Genova and Miss Gillian D. Hodge for their help in typing the manuscript, and to L.M. Dickens for his contribution to the understanding of the illustrations. Finally, I am indebted to W. Ambrose and H. Samelson for sharing their insights via courses, notes, and conversations.

\bigskip{\hfill\scshape N.J. Hicks}

{\itshape
\noindent Ann Arbor, Michigan\\
May 1964
}

\end{document}